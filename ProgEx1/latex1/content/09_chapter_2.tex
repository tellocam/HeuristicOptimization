\section{Questions}

\textcolor{red}{
9. Perform some manual tuning of relevant algorithmic parameters to find sensible parameter settings
for the final experiments. Relevant parameters may be related to the degree of randomization,
neighborhood structure sizes, probabilities for the random step function in composite neighborhood
structures, the cooling schedule, the tabu list length and its variation, etc. Report the impact of a
number of different settings on the solution quality of a selected meaningful subset of instances.
11. Run experiments and compare all your algorithms on the instances provided in TUWEL:\\
(a) deterministic and randomized construction heuristic and GRASP\\
(b) Use the solution of the deterministic construction heuristic to test the other implementations:\\
i. Local search for at least three selected (possibly composite) neighborhood structures using
each of the three step functions (i.e., at least nine different algorithm variants).\\
ii. VND\\
iii. GVNS, SA, or TS\\
12. Write a report containing the description of your algorithms, the experimental results and what
you conclude from them; see the general information document for more details.}

\pagebreak

\section{Questions to Consider during developtment}

\textcolor{red}{
• Ad 3 and 4: Does randomization and iterated application improve the generated solutions?\\
• What parameters do you use and which values do you chose for them?\\
• How does incremental evaluation work for your neighborhood structures?\\
• What is the time complexity to fully search one neighborhood of your neighborhood structures?\\
• VND: Does the order of your neighborhood structures affect the solution quality?}