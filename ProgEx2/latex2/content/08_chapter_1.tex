\section{Introduction}

\begin{proof*}
    let $X \in [C^{0,1}(\bar{\Omega})]^2$ with $X \rvert_{\Gamma_{\infty}} = 0$ be a given vectorfield. \\
    Set $\mathrm{T}_t(.) := \mathrm{id} + tX $ , with $t \in \mathbb{R}$ and $\Omega_t := \mathrm{T}_t(\Omega)$, where 
    $(\mathbf{u}_t, p_t)$ solve this and $\Omega$ is replaced by $\Omega_t$ s.t. \\
    $p_t \in \mathrm{L}^2(\Omega_t), \int_{\Omega_t} p_t \, \mathrm{dx} = 0 $ and $\mathrm{u}_t \in [H^1(\Omega_t)]^2$. Then there holds 
    \begin{align}
        \int_{\Omega_t} \mathrm{D} \mathbf{u}_t : \mathrm{D} \mathbf{\mathbf{v}} + \mathrm{div}(\mathbf{v}) \, p_t + \mathrm{div}(\mathbf{u}_t) \, q \, dx = \, 0 \quad \forall (v,q)
        \in  [H^1_0(\Omega_t)]^d \times L^2(\Omega_t).
    \end{align}
    Introduction of change of variables shows that $(\mathbf{u}^t, p^t) := (\mathbf{u}_t \circ \mathrm{T}_t, p_t \circ \mathrm{T}_t)$ satisfy
    \begin{equation}
    \begin{aligned}\label{trafo_weak_stokes}
        &\int_\Omega \mathrm{det}(\mathrm{DT}_t)
        \left( \mathrm{DT}_t^{-1} \mathrm{D}\mathbf{u}^t:\mathrm{DT}_t^{-1} \mathrm{D}\mathbf{v} -p\, \mathrm{tr}(\mathrm{D}\mathbf{v}\mathrm{DT}_t^{-1})  +
        q \, \mathrm{tr}(\mathrm{D}\mathbf{u}\mathrm{DT}_t^{-1}) \right) \mathrm{dx}, \\
        & \quad \quad \quad \quad \quad \quad \quad \quad \quad \ \forall (v,q) \in [H^1(\Omega)]^2 \times L^2(\Omega),
    \end{aligned}
    \end{equation}
    Used in equation (\ref{trafo_weak_stokes})
    \begin{align*}
        \mathrm{D}\mathbf{v}\circ\mathrm{T}_t &= \mathrm{D}(\mathbf{v}\circ\mathrm{T}_t), \\
        \mathrm{div}(\mathbf{v}) &= \mathrm{tr} \left( \mathrm{D}(\mathbf{v} \circ \mathrm{T}_t)(\mathrm{DT}_t^{-1}) \right).
    \end{align*}
      
    The functional $J(\Omega,\mathbf{u})$ is now reduced to the functional $J(\Omega)$, since the change of the quantities $(\mathbf{u},p)$
    is taken into account by the transformation theorem. The minimum of (\ref{energy_dissipation_equation})
    satisfies the saddlepoint problem . It can be obtained with the Lagrange Multiplier method, 
    see Faustmann . The corresponding Lagrangian which can be used to minimize is
    \begin{equation}\label{parametrized_lagrangian}
     \begin{aligned}
        \mathcal{L}(t,\mathbf{v}, q) = \frac{1}{2}& \int_{\Omega} \mathrm{det}(\mathrm{DT}_t) \mathrm{D} \mathbf{v}(\mathrm{DT}_t)^{-1} :
        \mathrm{D} \mathbf{v}(\mathrm{DT}_t)^{-1} \, \mathrm{dx}, \\
        &- \int_{\Omega}\mathrm{det}(\mathrm{DT}_t)q \, \mathrm{tr} \left( \mathrm{D}\mathbf{v}(\mathrm{DT}_t)^{-1} \right).
    \end{aligned}
    \end{equation}
    To find the shape derivative, one can now derive this parametrized Lagrangian, for details on the derivation of parametrized Lagrangians, 
    see K. Ito et. al. . With the derivative of the Lagrangian obtained, it holds true that
    \begin{align}
        \mathrm{d}J(\Omega)(X) =\ \frac{\mathrm{d}}{\mathrm{d}t} \mathcal{L}(t, \mathbf{u}^t, 0)\big\rvert_{t=0}  =
        \frac{\partial}{\partial t}\mathcal{L}(0,\mathbf{u},p) = \int_{\Omega} \mathrm{S}_1 : \mathrm{D}X \, \mathrm{dx}.
    \end{align}
    \qed
    \end{proof*}